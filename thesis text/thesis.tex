\documentclass{article}
%\documentclass[journal]{IEEEtran}
%\documentclass{report}
%\documentclass{acta}

\usepackage{graphicx}

\begin{document}

\title{Constructing three-dimensional (3D) polycrystalline models of FCC crystal for numerical modeling and simulation with multibody potential}
\author{Rustam Akhunov}

\maketitle

\tableofcontents

\begin{abstract}
The three-dimensional (3D) polycrystalline models of FCC crystal with the log-normal grain size distribution are constructed by constrained Voronoi tessellation. For achieving needed distribution of grain sizes and grain orientation we used Genetic Algorithm (GA) with Least Square (LS). We used molecular-dynamics method in LAMMPS software to obtain the relaxed polycrystal. For the relaxation process and for simulation we used multibody potential.
\end{abstract}


\section{Introduction}

There are known a lot of modern application of polycristalline materials. Polycristallines are used for tritium generation \cite{john99}, automotive, healthcare and non-destructive testing \cite{pardo11}, solar cells \cite{schrop98}, semiconducting elements \cite{harb85} etc. With increasing demand of new properties of new materials increases also need for the modern methods of polycristalline simulation \cite{shen15}. And this work is one more attempt to deal with the simulation of polycryrstallines in molecular dynamics environment such as LAMMPS \cite{plimp95, plimp00}

There are numbers of experiments and theoretical research for the intrinsic structure, effective thermal conductivity, lattice dynamical, optical and thermodynamic properties of polycristalline \cite{shen15}. But still many characteristics have not been researched sufficiently.

We know few methods for generating digital nanostructured materials for use in the simulation: Poisson–Voronoi tessellation (PVT) \cite{wear86}, Monodispersive grain size (MGS) \cite{wang96}, Laguerre–Voronoi tessellation (LVT) \cite{fan04}, Johnson–Mehl (JM) model \cite{okabe00} etc. In this work we used Voronoi tesselation model for the reason of its good solution for the complex sctructural nanocrystals and also convinient interface for the genetic algorithm, that we used during our polycristalline generation. It is also generally accepted that many single-phase fully dense nanocrystals are described by a log-normal grain size distribution. And we took this distribution as a reference for polycristalline generation.

\section{Results and Discussion}


\subsection{Construction of an initial nanocrystal}

As we know, the microstructure plays an important role in determining various properties of nanocrystals. This work uses Li4SiO4 with monoclinic structure as a model material and provides a new method to simulate the atomic structural properties of the sample. The algorithm we developed is described as follows:

(1) Construct an orthogonal box with volume V1, and randomly generate N points or grains within the box;

(2) The Voronoi tessellation using these center points is built and the size-distribution  and the penalty function W2 () are computed. For improving the quality of solutions and reducing execution time, we introduce the genetic algorithm method combined with the least square method instead of the inverse Monte Carlo method of Gross et al.15. and the Genetic Algorithm of Tomoaki et al.16. The entire process repeats many times until W2 approaches the optimization criterion at W2 <10-6. Finally, we need retain these ultimate points which meet the requirements of W2 <10-6, and rebuild Voronoi polyhedron cells around each of these ultimate points.

(3) When using the atoms to fill the orthogonal box, we select the volume V2, according to a certain proportion, from the entire box (V1) to fill atoms in order to observe the structure of nanomaterial more clearly (see Fig. 1a). Generate a big enough supercell of Li4SiO4 with the volume V3 which should contain the circumscribed sphere of the volume V2 (V3 > V2); Move the supercell and make the center of the supercell and the center of the new box with the volume V2 overlap;

Figure 1
Figure 1
(a) The constructed box. (b) The unrelaxed atomic structure of a Li4SiO4 nanocrystal. (The color of Li, Si and O in the crystal atoms is red, yellow, and green, respectively. The color of Li, Si and O at the boundary atoms is blue, pink, and white, respectively; rLi > rSi > rO).

Full size image
(4) Select the supercell’s rotation axis randomly and rotate the supercell according to arbitrary rotation angle. Fill one of Voronoi polyhedron cells in the new box with qualified atoms in the supercell. The qualified atoms need meet the following conditions18: The distance between the atom and the edge of a polyhedron is smaller than BDV (Boundary Displacement Variable) = βr0, (β ≥ 0, r0 is the atomic radius), while the nearest distance between two atoms on the two adjacent polyhedra is set no less than the diameter of the atom according to the idea of the hard-sphere model.

(5) The fourth step is repeated until all polyhedra in the new box are filled.

(6) Delete the excess atoms on the polyhedron boundary in order to keep the whole nanocrystalline material showing the charge neutrality (see Fig. 1b). At the same time, three-dimensional periodic boundary condition for the large-block is adopted, which will make the block effectively infinite and eliminate the influence of the boundary on the simulation results.

\subsection{Mean Grain Size Distribution}

Among the simulation variables, the mean grain size (D) is the most important one because it direct controls the volume fraction of the boundary atoms. By using the new stochastic search and geometric computation, the mean grain size distribution of the 3D nanocrystalline model of Li4SiO4 with 1000 grains is obtained (see Fig. 2). The targeted grain size distribution is a log-normal grain size distribution with a variance of (0.35)2, the same as reported in Refs15,16.

Figure 2
Figure 2
(a) The mean grain size distribution of the 3D model of nanomaterial with 1000 grains at the beginning of the algorithm is represented by the histogram. (b) The calculated mean grain size distribution of the 3D model of nanomaterial with 1000 grains after 10000 steps of evolution is represented by the histogram. The red line indicates the desired log-normal size distribution with σ = 0.35 and the fitted size-distribution of the histogram has σ = 0.34.

Full size image
During evolution process, we improve the algorithm by combining Genetic Algorithm (GA) and Least Square (LS) method to produce better quality results in smallest amount of time. The GA is a kind of global optimal searching algorithm based on Darwin’s nature evolution theory and Mendel’s genetics and mutation theory. The conceptualisation of GA in current form is generally attributed to Holland19. Compared with any other optimization algorithms, the outstanding excellences of GA are the capability of global optimization, strong robustness and inherent parallelism. However, GA has the shortcoming that it is easy to trap at local minima and time-consuming to find an optimal solution20,21. Therefore, we introduce the Least Squares (LS) method in this work. It is known to all that LS is a very good mathematical optimization technique which is usually used to find the best match solution by minimizing the error value between actual data and calculated data. Moreover, the volume of each Voronoi Cell is calculated by Voro++which is a software library for carrying out three-dimensional computations of the Voronoi tessellation22.

In Fig. 2, it can be easily seen that the mean grain size distribution of the sample obey the log-normal grain size distribution almost perfectly after 10000 steps of evolution, which is better than the results in Refs15,16. In this study, we adopted a population size of 32, the same as the number of population in Ref16. The computation time per step was 3.7 s with 6 processors, which has a higher efficiency than that of 3.9 s with 32 processors16. The mutation rate per center point is set at 0.1 for the first 500 generations, 0.01 for 500 ~ 1000 generations, and 0.001 for the rest. By applying the new method, we can give priority to a certain part of the mean grain size distribution function which matches worst with the standard log-normal distribution. During the study, we use a new fitness function F instead of the original fitness function W2 (or 1/W2) to help to escape local optimal solution efficiently. F is defined as:  ;( is an arbitrary constant and  > 0). Thus, when the value of W2 has the same value, the result will give priority to the influence of the value of . The smaller the value of  is, the better the value of F is. By utilizing the new fitness function F, we improve the quality of solutions and reduce execution time. Fig. 3 presents a 3D view of a polycrystalline structure with 1000 grains created by the new method.

\subsection{Relaxed Atomic Structure}

Computer simulation is a powerful method to obtain the numerical approximation of microstructural changes taking place in nuclear materials. In fact, in the process of deleting the excess atoms on the polyhedron boundary, some artificial defects and torsions at the grain boundary is inevitably introduced, such as, the existence of the vacancy, the mismatch of ionic bonds, tilt boundary, and so on. For investigating the atomic position change and eliminate the influence of the artificial defects and torsions at the grain boundary of nanostructured Li4SiO4, the relaxation procedure was performed with the LAMMPS code23, based on the molecular-dynamics (MD) theory.

In this work, the primitive cell of Li4SiO4 has a monoclinic structure with space group of P21/m (No.11)24. The atomic potential of Li4SiO4 used in this work, which is the pair potential function, is from the studies of Takahashi et al.25. In physics, a pair potential is a function that describes the potential energy of two interacting objects. Hence, we have adopted the same pair potential within crystal and among crystals. The pair potential functions are assumed to consist of simplified coulombic and repulsive terms: , where  and represent ionic charges, respectively, [Å] is the distance between the ions i and j, a and b are the parameters related to the radius and the compressibility of each ion, respectively,  is an arbitrary constant (taken to be 6.7472*10−11N). The parameters used in the calculation are given in Table 1.

Table 1: Potential parameters used in the MD calculations.
Full size table
To perform the simulations, the cutoff distance for the Takahashi’s potential rc was set to 10 Å. The coulombic term was evaluated using the Particle–Particle Particle-Mesh (PPPM) method27, while the repulsive term was described by a Buckingham potential. The Periodic Boundary Conditions (PBC) was employed. As the starting configuration, the simulation was run for 30,000 time steps (30 ps) in an NPT ensemble at zero external pressure to ensure that the system have adequate time to obtain a suitable equilibrium structures after relaxed process. In the process of MD run, the orthogonal box was flexible. The system was thermalized at 300 K. A time step of 0.001 ps was used in the MD runs. Finally, the system reached the equilibrium state because the fluctuation range of its temperature or energy was not very intense. Although this work has focused on Li4SiO4, the technique described here is general and it can be applied equally to any nanomaterial for which its atomic potential is available.

The unrelaxed and relaxed atomic structures of nanostructured Li4SiO4 are illustrated in Fig. 4. As we can see, in Fig. 4(a), the atoms are arranged in an orderly state, while the arrangement of atoms is in the messy state in Fig. 4(b). Firstly, the phenomena may be caused by the change of the positions of the atoms which will move to the positions with minimum energy in order to achieve the steady state after the relaxation process. Secondly, the shape of the nanocrystal may also affect the simulation results as mentioned in Barnard et al.‘s work28,29.

Figure 4
Figure 4
(a) The unrelaxed initial structure of a Li4SiO4 nanocrystal. (b) The relaxed atomic structure of a Li4SiO4 nanocrystal. (Mean grain size D=6.61 nm, BDV=0.5r0, where r0 is the radius of the Li atom).

Full size image
In Fig. 4(a), the whole system is in a metastable state. With the development of the relax process, it will experience the transition from metastable state to steady-state. Finally, the whole system will reach the equilibrium state (see Fig. 4b). After relaxation process, the calculated bond distances of Li-Li, Li-O and Si-O of Li4SiO4 within the crystal are about 2.373–2.619 Å, 1.865–2.478 Å and1.589–1.712 Å, which are very close to 2.385–2.595 Å for Li-Li, 1.863–2.457 Å for Li-O and 1.597–1.696 Å for Si-O in the perfect crystal of Li4SiO430, respectively. However, the bond distances between atoms at the grain boundary are not with certain regularity properties. By and large, although the atomic positions of nanostructured Li4SiO4 change after relaxation process, the bond distances between atoms within the crystal have good consistency with the uniform distributed crystal system.

In addition, the sample reveals a reduction in the atomic densities after relaxation process. In Table 2, the structural information of 3D model of the sample after relaxation process is given by the AtomEye31. It’s easy to observe that the stoichiometry is conserved and the ratio of Li: Si: O is still 4:1:4 in the final nanocrystal (see Table 2). The results also show that the average mass density of the sample is slightly lower than the experimental data of the perfect crystal32, which is a normal phenomenon because of the presence of the grain boundary atoms. Similarly, the phenomenon was discovered by Herr et al.33. using nanocrystalline Fe as the sample material.


\subsection{Boundary Component Proportion (BCP)}

In order to illustrate the validity of the numerical simulation and modeling, we calculate the boundary component proportion (BCP) of nanostructured Li4SiO4 after relaxation process. (Where BCP=Nb/Nt, Nb is the number of boundary atoms, Nt is the total number of atoms in the nanostructured Li4SiO4.)

In Fig. 5, we observe that the values of the boundary component proportion are very high (0.51–0.78). All the BCP’s are higher than 0.5, which means more than 50 percent of the atoms are situated in the boundary region. Compared the results of our calculations with the theoretical calculation of Chen18, it is of interest to note that our results display very good agreement with the values of Chen who mainly focused on the model material for the Fe nanocrystal. On the experimental side, Wallner et al.34. also observed that the mean atomic density in the interfaces was about 0.52 of the lattice density by utilizing the small angle scattering of neutrons and X-rays.

Figure 5
Figure 5
Boundary component proportion (BCP) of nanostructured Li4SiO4. The black and red line corresponds to the variable of BDV=0.5r0 and BDV=1.0r0, respectively.

Full size image
Furthermore, we observe that the values of the BCP present a declining trend as the mean grain size is raised (see Fig. 5). The change of the BCP in our calculations for the sample may be induced by the low boundary density and the enlarged interatomic spacing at the boundary region as the mean grain size of the sample is increased. The atomic structure of an interface depends on the orientation relationship between adjacent crystals and the boundary inclination18. With the elevation of the mean grain size of the sample, the interatomic spacing at the boundary region will enlarge due to the considerable misfit between adjacent crystals. A low boundary density and a high interatomic spacing will naturally lead to the decrease of the BCP in the nanostructured Li4SiO4.

\subsection{Density Reduction Proportion (DRP)}

For the purpose of studying the changes of the Li4SiO4 nanocrystal which is induced by a low boundary density, the reduced density proportion (RDP) of the Li4SiO4 nanocrystal are computed (Where RDP=Nt / Nb, Nb is the number of atoms in a perfect crystal with the same size as the nanostructured Li4SiO4).

As can be seen from the formula, the larger the value of DRP is, the higher the actual density of the sample is. Fig. 6 displays that the values of RDP vary between 0.65 and 0.87. We compare our calculated results with the recent theoretical and experimental data available, we find that our results is consistent with the calculated values by Chen (0.61–0.81)18 and the experimental results (0.6–0.9) under the different materials and experimental conditions35. That is to say, our method in this work is reliable and reasonable.

Figure 6
Figure 6
Reduced density proportion (RDP) of nanostructured Li4SiO4. The black and red line corresponds to the variable of BDV=0.5r0 and BDV=1.0r0, respectively.

Full size image
The above results may be attributed to the following reasons: On one hand, with the mean grain size increasing, the number of grains (N) in the same volume will decline. So the degree of misfit between the different grains will also decrease. On the other hand, an important effect is also worthy of our consideration that the results may be affected by the number of random displacement atoms at the grain boundary in the simulation. With increasing of the mean grain size, the number of the allowed random displacement atoms at the grain boundary will become small. Finally, when increasing the mean grain size of the sample, these factors will result in a lower value of BCP but a higher value of DRP as shown in Fig. 5 and Fig. 6, respectively.

By comparing our nanostructured 3D models as well as the nanocrystalline structures directly from experimental observations, such as electron backscatter diffraction (EBSD)17 or synchrotron radiation tomography36, our method is much more cost-effective and takes less time, which also proves its great potential value in applications for the development of nanocrystalline materials. The simulation demonstrates that the initial microstructure is very important because it depends sensitively on the methods used. Moreover, one innovative feature of the program developed is the possibility to generate 3D microstructures for the large-scale atomistic simulations and complex nanostructured modeling. Currently, the numerical modeling constructed using the new method is under way for heat transmission, mechanical properties, defect researches, surface microstructures, and transport properties of nanostructured Li4SiO4. The extension of the evolutional approach is capable of improving the efficiency of numerical computations largely. Hence, we hope that the method we developed here may potentially open a new path way for guiding the further theoretical researches and experimental explorations of nanocrystalline materials.

\section{Methods}

\section{Additional Information}

\section{Conclusion}


\begin{thebibliography}{9}

\bibitem{john99}
  Johnson C. E. J.
  \textit{Tritium behavior in lithium ceramics}
  Nucl. Mater
  1999

\bibitem{pardo11}
  Pardo, Lorena, Ricote, Jesús (Eds.)
  \textit{Multifunctional Polycrystalline Ferroelectric Materials}
  Springer
  2011

\bibitem{schrop98}
  Schropp, Ruud E. I. Zeman, Miro
  \textit{Amorphous and microcrystalline silicon solar cells : modeling, materials, and device technology}
  Boston (Mass.) : Kluwer academic
  1998.

\bibitem{harb85}
   Harbeke, G. (Ed.)
  \textit{Polycrystalline Semiconductors}
  Springer-Verlag Berlin Heidelberg
  1985

\bibitem{plimp95}
   S. Plimpton 
  \textit{Fast Parallel Algorithms for Short-Range Molecular Dynamics}
  J Comp Phys, 117, 1-19
  1995

\bibitem{plimp00}
   S. Plimpton 
  \textit{LAMMPS WWW Site}
  http://lammps.sandia.gov

\bibitem{shen15}
  Shen, Y. et al. 
  \textit{Constructing three-dimensional (3D) nanocrystalline models of Li4SiO4 for numerical modeling and simulation.}
  Sci. Rep. 5, 10698; doi: 10.1038/srep10698 
  2015

\bibitem{wang96}
  Wang J. et al. 
  \textit{Computer simulation of the structure and thermo-elastic properties of a model nanocrystalline material}
  Philos. Mag. A. 73, 517–555 
  1996

\bibitem{wear86}
  Weaire D. et al. 
  \textit{On the distribution of cell areas in a Voronoi network}
  Philos. Mag. B 53, 101–105 
  1986

\bibitem{fan04}
  Fan Z. G. et al. 
  \textit{Simulation of polycrystalline structure with Voronoi diagram in Laguerre geometry    based on random closed packing of spheres} 
  Comput. Mater. Sci. 29, 301–308 
  2004

\bibitem{okabe00}
Okabe A. et al. 
\textit{Spatial Tessellations-Concepts and Applications of Voronoi Diagrams}
Wiley, New York, 
2000

\bibitem{sivan98}
  S. N. Sivanandam , S. N. Deepa
  \textit{Introduction to Genetic Algorithms},
  Springer Publishing Company, Incorporated
  2007

\bibitem{melan98}
  Melanie Mitchell
  \textit{An Introduction to Genetic Algorithms}
  MIT Press, Cambridge, MA
 1998

\bibitem{suwas14}
  S. Suwas and R. K. Ray
  \textit{Crystallographic Texture of Materials}
  Springer-Verlag London 
  2014

\end{thebibliography}


\section{Acknowlegments}

\end{document}
